\documentclass[12pt]{article}
\pdfoutput=1
\usepackage{bm}% bold math
\usepackage{graphicx}
\graphicspath{}
%\usepackage{tabularx}
\usepackage{amsfonts}
\usepackage{amsmath}
\usepackage{amssymb}
\usepackage{amsbsy}
\usepackage{bbm}
\usepackage{hyperref}
\usepackage{mathdots}
\usepackage{booktabs}
\setlength{\heavyrulewidth}{1.5pt}
\setlength{\abovetopsep}{4pt}
\usepackage[margin=0.9in]{geometry}
\usepackage{nicefrac}
\usepackage{subcaption}
\newenvironment{psmallmatrix}
  {\left[\begin{matrix}}
  {\end{matrix}\right]}
  \usepackage{listings}
\usepackage{braket}
\usepackage[percent]{overpic}


\begin{document}

Hamiltonian:
\begin{equation}
\mathcal{H} = \beta H = \sum_{\alpha} K_\alpha S_\alpha \\
\end{equation}


Truncate to nearest neighbor (NN) and next nearest neighbor (NNN) spin interactions:
\begin{equation}
\mathcal{H} \approx K_1 S_1 + K_2 S_2 
= K_1 \sum_{\langle i, j \rangle} \sigma_i \sigma_j + K_2 \sum_{\langle \langle i, j \rangle \rangle} \sigma_i \sigma_j \\
\end{equation}

Known critical temperature: 
\begin{equation}
T_c = - \frac{2 J}{\ln (1+\sqrt{2})} \approx - 2.27 J
\end{equation}

Expected critical coupling:
\begin{equation}
K_c = \frac{J}{T_c} = -\frac{\ln (1+\sqrt{2})}{2} \approx -0.44
\end{equation}

Probability of spin flip:
\begin{equation}
P(\sigma_i \rightarrow -\sigma_i) = \exp(-\Delta \mathcal{E})\\
\end{equation}
\begin{equation}
\Delta \mathcal{E} 
= -2 \sigma_i  \left( K_1 \sum_{\langle j \rangle} \sigma_j 
+ K_2\sum_{\langle \langle j \rangle \rangle} \sigma_j \right)\\
\end{equation}

Magnetization:
\begin{equation}
\mathcal{M} 
= \sum_i \sigma_i
\end{equation}

Variances of energy and magnetization:
\begin{equation}
\sigma_\mathcal{E} ^2
=\langle  \mathcal{E} ^2 \rangle 
- \langle  \mathcal{E}  \rangle ^2
= \sum_{\alpha \beta} K_\alpha K_\beta \big( \langle S_\alpha S_\beta \rangle
- \langle S_\alpha  \rangle 
\langle  S_\beta \rangle \big)\\
\end{equation}
\begin{equation}
\sigma_{\mathcal{M}, \alpha \beta} ^2
=\langle  \mathcal{M} ^2 \rangle 
- \langle  \mathcal{M}  \rangle ^2
=  \big( \langle S_\alpha S_\beta \rangle
- \langle S_\alpha  \rangle 
\langle  S_\beta \rangle \big)\\
\end{equation}


Linearized transformation matrix:
\begin{equation}
T_{\alpha, \beta}^{(n)} = \frac{\partial K_\alpha^{(n+1)}}{\partial K_\beta^{(n)}}\\
\end{equation}

Chain rule and invert:
\begin{equation}
\frac{\partial \langle S_\gamma^{(n+1)} \rangle}{\partial K_\beta^{(n)}}
= \frac{\partial \langle S_\gamma^{(n+1)} \rangle}{\partial K_\alpha^{(n+1)}}
\frac{\partial K_\alpha^{(n+1)}}{\partial K_\beta^{(n)}} 
\ \ \Longrightarrow \ \ 
T_{\alpha, \beta}^{(n)} 
= \left[ \frac{\partial \langle S_\gamma^{(n+1)} \rangle}{\partial K_\alpha^{(n+1)}} \right]^{-1} \frac{\partial \langle S_\gamma^{(n+1)} \rangle}{\partial K_\beta^{(n)}} \\ 
\end{equation}

where
\begin{equation}
\frac{\partial \langle S_\gamma^{(n+1)} \rangle}{\partial K_\alpha^{(n+1)}}  
=  \Big\langle S_\gamma^{(n+1)} S_\alpha^{(n+1)} \Big\rangle
- \Big\langle S_\gamma^{(n+1)}  \Big\rangle 
\Big\langle  S_\alpha^{(n+1)}  \Big\rangle
\end{equation}
\begin{equation}
\frac{\partial \langle S_\gamma^{(n+1)} \rangle}{\partial K_\beta^{(n)}}  
=  \Big\langle S_\gamma^{(n+1)} S_\beta^{(n)} \Big\rangle
- \Big\langle S_\gamma^{(n+1)}  \Big\rangle 
\Big\langle  S_\beta^{(n)}  \Big\rangle
\end{equation}
\newpage

\begin{center}
\begin{tabular}{|c|c|c|c|c|}
\hline
n & N = 16 & N = 32 & N = 64 & N = 128\\
\hline
0 & -0.4420811188 & -0.4417429398& -0.4415445759 & -0.4413292658\\
1 & -0.4404791358 & -0.4404577112 & -0.4404935597& -0.4404266727\\
2 & -0.4401272033 & -0.4402638848 & -0.4403468642& -0.4403131062\\
3 & - & -0.4402975527& -0.4403307834 & -0.4403016019\\
4 & - &  -& -0.4403116430 &-0.4402985574 \\
5 & - &  -& -& -0.4402807798\\
\hline
\end{tabular}
\end{center}


\newpage  

\begin{center}
\begin{tabular}{|c|c|c|c|c|}
\hline
n & N = 16 @ $K_1^c(N)$ & N = 16 @ $K_1^c$ & N = 32 @ $K_1^c(N)$ & N = 32 @ $K_1^c$\\
\hline
0 & 1.0366474702 & 1.0371936855 & 1.0359594271 & 1.0364461518\\
1 & 1.0055084088 & 1.0058615220 & 1.0030411491& 1.0024276306\\
2 & 0.9850458988 & 0.9832937022  & 1.0042751683  & 1.0040962187\\
3 & - & -  & 0.9833658883  & 0.9817702370\\
\hline
\end{tabular}
\end{center}

\newpage  

\begin{center}
\begin{tabular}{|c|c|c|c|c|}
\hline
n & N = 64 @ $K_1^c(N)$ & N = 64 @ $K_1^c$ & N = 128 @ $K_1^c(N)$ & N = 128 @ $K_1^c$\\
\hline
0 & 1.0340840722& 1.0356603405 & 1.0343762634 & 1.0349921920\\
1 & 1.0013550363 & 1.0025377191 & 1.0018397044& 1.0006737634\\
2 & 1.0021687106 & 1.0029573471  & 1.0019303651  & 1.0022232598\\
3 & 1.0029619937 & 1.0039006140  & 1.0025651221  & 1.0023411428\\
4 & 0.9820802731 & 0.9819506190  & 1.0035789853  & 1.0040042374\\
5 & - & -  & 0.9814148379  & 0.9819137809   \\
\hline
\end{tabular}
\end{center}



\end{document}